\documentclass{article}
\usepackage{fullpage}
\usepackage{enumitem}
\usepackage{verbatim}
\usepackage{listings}
\usepackage{color}
\usepackage{hyperref}
\usepackage[usenames,dvipsnames,svgnames,table]{xcolor}

\begin{document}
\lstset{language=C++,captionpos=b,basicstyle=\small\sffamily,
  columns=fullflexible, xleftmargin=.25in, xrightmargin=.25in,
  resetmargins=true}
\newcommand{\choice}[1]{\textbf{Choice} #1}
\newcommand{\reason}[1]{\textbf{Reason} #1}
\newcommand{\term}[1]{\textbf{#1}}

\newcommand{\code}[1]{\lstinline!#1!}
\newcommand{\citem}[1]{\item{\code{#1}}}

\newcommand{\lclass}[1]
{
\noindent\makebox[\linewidth]{\rule{\textwidth}{1pt}} 

\textsc{Class} \code{#1}
}
\newcommand{\class}[1]
{
\textsc{Class} \code{#1}
}
\newcommand{\eoclass}
{
\noindent\makebox[\linewidth]{\rule{\textwidth}{1pt}}
}

\newcommand{\desc}[1]{\textbf{Description} #1}
\newcommand{\mems}[1]{\textbf{Data Members} \\ #1}


\newcommand{\cons}[1]{\textbf{Construction} \\ #1}
\newcommand{\oper}[1]{\textbf{Operators} \\ #1}
\newcommand{\memfns}[1]{\textbf{Member Functions} \\ #1}

\title{\textbf{Serialization Library}}
\maketitle
\begin{center}
\begin{tabular}{c c c}
Serialization Library \\
Design Document
\end{tabular}
\end{center}

\vspace{1.5cm}

\tableofcontents

\section{Bird's-Eye View}
Thinking of a possible design for a serialization library in C++
immediately suggests that templates are going to be integral if any
hopes of genericity are to be realized. Our proposed design is no
exception, and due to the necessity of templating most of the
functions we provide, inheritance is forced to play a smaller part.

At a high level, the library uses \code{StreamWriter}s and
\code{StreamReader}s to perform the serialization and the
deserialization, respectively. They are wrapped around \code{ostream}
and \code{istream} objects. Here on in this document, we refer to
`Reader/Writer' when we mean objects which may be of type either
\code{StreamReader} or \code{StreamWriter}.

Strictly speaking, \code{StreamWriter} and \code{StreamReader} are
abstract classes, and the classes which do the real work are derived
from these two, implementing the core functions in their own
way. \code{TextStreamWriter} and \code{TextStreamReader} form one set
of complementary implementations provided. 

The \code{<<} and \code{>>} operators are implemented for
\code{StreamWriter} and \code{StreamReader} objects, and when used on
a given `serializable' object, perform the implementation-specific
`saving' and `loading' of data.

\code{save} and \code{load} member functions are expected to be
defined in the implementation for each type of object the user
attempts to serialize or deserialize. It is desirable to generate an
error at compile-time when possible to inform the user of any missing
functions.

\section{Brief Requirements}
\begin{enumerate}
\item Provide the user with a familiar syntax, preferably \code{<<} and \code{>>}.
\item \textit{Mandatory:} Support serializing (and, implicitly,
  deserializing) of all fundamental types.
\item \textit{Mandatory:} Support serializing of common standard
  library containers without extra user intervention.
\item \textit{Mandatory:} Support serializing of user-defined classes.
\item \textit{Optional:} Support serializing pointers and references
  (but make it easy to extend the library to support them without
  design changes).
\item \textit{Optional:} Handle polymorphic objects correctly.
\item Provide a portable text-based implementation. (The working
  definition of `portable' throughout was ``should work on at least
  one Linux, Windows, and Mac without changing anything.'')
\end{enumerate}

\section{Limitations in Requirements}
\begin{enumerate}
\item We do not enforce portability since it is in some cases
  (eg. binary) quite difficult to implement, and not always critical.
\item Pointers for arbitrary types are not supported, though it is not
  foreseen to be difficult once the object-tracking mechanism is
  implemented.
\end{enumerate}

\section{Design}

\subsection{Serializability}
We refer to an object of type \code{T} as being \textbf{serializable} if:
\begin{enumerate}
\item \code{T} is a fundamental type: any of the standard
  boolean, character, integer, or floating point types.
\item \code{T} is of type \code{std::string}.
\item \code{T} is of class type with the \code{serialize} (and
  \code{deserialize}, for the deserialization process) function for
  \code{T} defined.
\item \code{T} is a pointer to a polymorphic object which is
  serializable and has a zero-argument constructor.
\item \code{T} is an array of serializable objects.
\end{enumerate}

All implementations must be able to serialize any of these
serializable objects.

The definition gives a clear view of what types are currently
supported by the library. Generic pointers are not yet supported, and
polymorphic objects must have a zero-argument constructor for
deserialization to work.

\subsection{Interface}

\lclass{StreamWriter}
\begin{description}
\item \cons{ Specify an open \code{std::ostream} object.}
\item \mems{ \code{stream}: Pointer to the \code{std::ostream} object.}
\end{description}

\class{StreamReader}
\begin{description}
\item \cons{ Specify an open \code{std::istream} object.}
\item \mems{ \code{stream}: Pointer to the \code{std::istream} object.}
\end{description}
\eoclass


Two main base classes, \code{StreamWriter} and \code{StreamWriter},
exist, which provide an interface for the concrete Reader/Writer
classes. Ideally, they would also provide some standard implementation
helpers for the derived classes, but there was not found to be much
genuinely common between all possible derived classes except possibly
the \code{istream} or \code{ostream} objects.

The \code{<<} and \code{>>} operators need to be defined for
\code{StreamWriter} and \code{StreamReader} derived classes, and need
to work with any given type. In effect, for example considering
\code{TextStreamReader} and \code{TextStreamWriter} as the derived
classes, the desired syntax is:

\begin{lstlisting}
TextStreamWriter writer(out_stream);
writer<<any_object_to_write;

TextStreamReader reader(in_stream);
reader>>any_object_to_read;
\end{lstlisting}

They are the major (and in many cases, the only) functions the user
calls from the derived class objects.


Since the object on the right side can be of any type, the operators
must be templated on this object.  

Also, thinking ahead to what might happen in the body of the
operators, it appears it is totally up to how \code{TextStreamWriter}
and \code{TextStreamReader} implement the operator for themselves, and
as such are free to do anything, calling any member functions and so
making the design inconsistent.

We see this to be the only place to enforce design constraints for
\code{TextStreamWriter} and \code{TextStreamReader} by defining the
operators at a global level, independent of the derived classes.

The operators are templated on the object type, and the body of the
operator will involve calling another derived class-specific template
function which performs the actual saving or loading. Virtual
functions would be normally used for this, but due to the template
requirement, we must make the operator a non-member function.

To restrict each operator's applicability, \code{std::enable_if} is
used.

\begin{lstlisting}
template <typename Reader, typename T>
typename std::enable_if<std::is_base_of<StreamReader, Reader>::value, Reader&>::type
operator>>(Reader & reader, T & T_data)
{
  reader.load(T_data);          // derived class must implement templated load(T&)
  deserialize(reader, T_data);
  return reader;
}

template <typename Writer, typename T>
typename std::enable_if <std::is_base_of<StreamWriter, Writer>::value, Writer&>::type
operator<<(Writer & writer, const T & T_data)
{
  writer.save(T_data);          // derived class must implement templated save(const T&)
  serialize(writer, T_data);
  return writer;
}
\end{lstlisting}

We have not lost out by not using virtual functions - the template
functions are equally easy to specialize, and provide a good design
enforcement mechanism, without any overhead. However, unhelpful
compiler error messages become a problem.

In an early version the derived classes were supposed to implement
these operators as member functions, and though that can be made to
work, is a source of subtle errors - for example, forgetting to return
\code{*this}. These problems are eliminated by defining them as a
couple of template functions globally.

\subsection{Inside the Derived Reader/Writer Classes}
\lclass{TextStreamWriter}
\begin{description}
\item \cons { Specify an open \code{std::ostream} object.}
\item \textbf{Member Functions}

  Variants of \code{save} are the only public member functions to be
  defined. The general structure in the current implementation is:

\begin{lstlisting}
template <typename T>
typename std::enable_if<std::is_fundamental<T>::value>::type
save(const T & T_data)
{
  write_type(T_data);
  write_data(T_data); // we know how to write fundamental data ourselves
}

template <typename T>
typename std::enable_if<std::is_class<T>::value>::type
save(const T & T_data)
{
  write_type(T_data);
  // we don't know how to write class data by ourselves
}
\end{lstlisting}

\code{save} is defined for other groups of types similarly.
\end{description}

\class{TextStreamReader}
\begin{description}
\item \cons{ Specify an open \code{std::istream} object.}
\item \textbf{Member Functions}
  Variants of \code{load} are the only public member functions to be
  defined. The general structure in the current implementation is:


\begin{lstlisting}
  template <typename T>
  typename std::enable_if<!std::is_class<T>::value>::type
  load(T & T_data)
  {
    read_and_check_types(T_data);
    read_data(T_data); // can read data ourselves if not class type
  }

  template <typename T>
  typename std::enable_if<std::is_class<T>::value>::type
  load(T & T_data)
  {
    read_and_check_types(T_data);
    // don't know which class members to read data into
  }
\end{lstlisting}

\end{description}
\eoclass

Specializing \code{load} and \code{save} for each group of types is
done at the derived class level. The implementation decides whether to
write type information or not, and if so, for which groups of types it
should be written.

The \code{save} and \code{load} functions must be `in sync' with each
other, and for a while there was a temptation to write
\code{TextStreamWriter} and \code{TextStreamReader} together in the
same file to make this easier. This was dropped as it turned out to be
simple enough to keep them consistent even across different files if
one function was changed at a time.

It can be seen from the code examples above that for fundamental
types, \code{TextStreamWriter} can write data completely without any
extra user help. However, for classes such as:

\begin{lstlisting}
class Cls
{
  int first_mem_int;
  double second_mem_double;
  string third_mem_string;
  int fourth_mem_int;
};
\end{lstlisting}

there is no general way to serialize each member of the class and
deserialize them back into the same members correctly. This can be
done in special cases such as for binary serializers which may copy
the entire representation from the memory, but even that approach is
not completely fool-proof. In most cases, not all members are desired
to be serialized, making an entire memory copy inefficient. Also,
doing a memory copy would copy the addresses held by each member
pointer but that would result in incorrect behaviour on
deserialization. PODs seem to be more friendly towards the memory
copy, but not classes in general.

For other serializers, there does not appear to be a reasonable way to
do it (even for PODs) without some help from the user.

\subsection{Dealing with Classes}
Having become convinced that user intervention is required for
classes, the design of the method of intervention is the next step.
The details the user must give to the library are:
\begin{enumerate}
\item Exactly which members (and in what order) of the class to
  serialize.
\item While deserializing, exactly which members (and in what order)
  of the class should be read into.
\item If the class is a derived class, whether the base object (and
  its relative order in (de)serialization) must also be (de)serialized.
\item Importantly, this method should be independent of the
  implementation used, eg., text/binary implementations of the
  serializer.
\end{enumerate}

The generic behaviour required makes a good case for this
`intervention' method to be a template function (templated on the
Reader/Writer). Also considered but eventually not implemented was
specifying this as an ordered `list' of members of the class somehow,
but making such a list possible is not very straightforward and takes
away some power from the user as compared to if this was a function.

An example of the way the template user-defined functions might be
implemented for \code{Cls} is:
\begin{lstlisting}
template <class Writer>
void serialize(Writer w, const Cls & cls_object)
{
  w<<cls_object.first_mem_int;
  w<<cls_object.second_mem_double;
  w<<cls_object.third_mem_string;
  w<<cls_object.fourth_mem_string;
}

template <class Reader>
void deserialize(Reader r, Cls & cls_object)
{
  r>>cls_object.first_mem_int;
  r>>cls_object.second_mem_double;
  r>>cls_object.third_mem_string;
  r>>cls_object.fourth_mem_string;
}

int main()
{
  // truncated
  Cls cls_object;
  TextStreamWriter w(stream); // stream is an ostream
  w<<cls_object; // << defined for classes such that it calls serialize()
}
\end{lstlisting}

The \code{w<<cls_object;} statement in \code{main()} would invoke the
\code{serialize} function for the class, and this in turn invokes the
\code{<<} operator for each of the members specified. This goes on
recursively upto the fundamental type objects which form the leaf
nodes and are serialized since as a requirement, the library must know
how to serialize fundamental types.

There is a flaw in this particular example -- the members have by
default the \textbf{private} access specifier, and \code{serialize}
and \code{deserialize}, being free functions, cannot access them
without another user intervention.

\subsubsection{Access to Class Members}
The objective is to get \code{serialize} and \code{deserialize} to be
able to access non-public members of the class. Here is another
decision point:
\begin{enumerate}
\item Should \code{serialize} and \code{deserialize} be declared as
  \code{friend} functions of the user-defined class?
\item Should \code{serialize} and \code{deserialize} be defined as
  member functions of the user-defined class?
\end{enumerate}

Both these options require some kind of change to the user's class --
a \code{friend} declaration or a new member function definition, and
are functionally equivalent.

Writing them as member functions results in more compact code. But
they should themselves be made \code{private} so that they do not form
part of the interface of the object. Only the serialization library's
operators should have access to these member \code{serialize} and
\code{deserialize} functions.

To enable this kind of behaviour, we would need to make the
serialization library `friends' with the user's class. This in itself
involves a \code{friend} declaration.

On the other hand, making \code{serialize} and \code{deserialize} free
functions, with each of them declared \code{friend}, makes the code more
verbose but leads to a simpler design.

The best option would be to support both these methods. The library
currently supports the free \code{friend} function style, but adding
support for the other method is not too difficult:

\begin{lstlisting}
// As defined earlier
template <typename Writer, typename T>
typename std::enable_if <std::is_base_of<StreamWriter, Writer>::value, Writer&>::type
operator<<(Writer & writer, const T & T_data)
{
  writer.save(T_data);          // derived class must implement templated save(const T&)
  serialize(writer, T_data);
  return writer;
}

// Implementation of default serialize for classes
template <typename Writer, typename T>
typename std::enable_if < std::is_base_of<StreamWriter, Writer>::value &&
			  std::is_class<T>::value >::type
serialize(Writer & writer, const T & class_obj)
{
  class_obj.serialize(writer);
}

// If user defines a free serialize, it gets called preferentially during overload resolution
// else the user must define a member serialize which is called by default
\end{lstlisting}

This enables the member-\code{serialize} syntax, but the problem with
access is not yet resolved, since our default \code{serialize} must be
able to call the class's (private) \code{serialize} member function. A
workaround would be to declare this default \code{serialize} function
to be a friend of the user's class. This is not much cleaner compared
to the free-\code{serialize} technique, and so it was decided to stick
with defining \code{serialize} as a free function.

Furthermore, if the user does not need to provide access to non-public
members, the free \code{serialize} function approach does not require
any change to the class definition, and so can be used for classes
whose code the user cannot change.

\subsubsection{Serializing Classes Belonging to an Inheritance Hierarchy}

When serializing a derived class, the approach given above fails to
serialize the base object's members unless it is explicitly instructed
to do so.

There is a natural solution to this -- calling the \code{<<} operator
on the derived object cast to the base type invokes the serialization
of the base object.

This means the user must define the \code{serialize} function for all
base classes if the base objects are to be serialized.

\begin{lstlisting}
class DerivedCls: public Cls
{
  // dummy derived class
  char mem_char;
};

template <typename Writer>
void serialize(Writer & w, const DerivedCls & derived_object)
{
  w<<(static_cast<const Cls>(derived_object)); // serialize base class
  w<<mem_char;
}
\end{lstlisting}

The same can be extended to \code{deserialize}, only without the
\code{const} specifier.

\subsection{Arrays, Pointers, and Polymorphic Objects}

Having the basic mechanism for serializing fundamental types and
classes in place, extending the library to handle other variations is
mostly a matter of defining a template specialization for each type.

\subsubsection{Arrays}

The \code{type_traits} header includes a trait class (\code{extent})
which can be used to correctly get any dimension of a given
array. Prior to using this, the library used \code{sizeof}
comparisons, but on testing for multidimensional arrays, the results
were inaccurate.

Arrays are supported by the library, but not pointers which are to be
`interpreted' as dynamic arrays. 

\begin{lstlisting}
int matrix[][3] = {{1,2,100}, {3,4,101}, {5,6,102}}; // okay, serializes
int *y = new int[3]; // not supported; compile-time error on attempted serialization
\end{lstlisting}

The array may be of any serializable type.

\subsubsection{Pointers}

Pointers to arbitrary types are not supported. This is due to the
inherent ambiguity on whether the pointer represents a dynamic array
or points to a single object.

Support for \code{char*} pointers is implemented, but on the
assumption that they are null-terminated C-style strings.

A strategy for handling pointers in future versions is:
\begin{enumerate}
\item Allow serialization of pointers by assuming the pointer points
  to one object.
\item To serialize dynamic arrays such as \code{int* arr = new
    int[10];}, the user will have to use a syntax like \code{for (int
    i=0; i<10; ++i) writer<<&arr[i];}.
\item Implement a pointer swizzling and deswizzling system. If we make
  (severe) assumptions on the relations between all the pointers, we
  can describe a simple swizzling/deswizzling system.
\end{enumerate}

As a simplistic description, the Writer/Reader objects would store a
list of addresses of each object serialized/deserialized so far. When
a pointer is to be serialized, the Writer looks up this list to check
if the address being pointed to is present, and if so, at which
index. If a valid index is obtained, the index is stored in the stream
(along with a tag to indicate that this is a pointer reference, and
not a numeric type). Else, the value of the object is serialized and
the address is added to the list of known addresses.

At the other end, the Reader also maintains a corresponding list of
addresses of already deserialized objects. If a pointer reference is
seen, then instead of 

\end{document}