\documentclass{article}
\usepackage{fullpage}
\usepackage{enumitem}
\usepackage{verbatim}
\usepackage{listings}
\usepackage{color}
\usepackage{hyperref}
\usepackage[usenames,dvipsnames,svgnames,table]{xcolor}

\begin{document}
\title{\textbf{CSCE689: Project: Serialization Library in C++11}}
\maketitle

\begin{center}

%\vspace{0.5cm}
\begin{tabular}{c c c}
Raghdah Al-Shaikhli & raghdah@tamu.edu & UIN: 322000076 \\
Shreyas Vinayakumar & vshreyas@tamu.edu & UIN: 521007089 \\
Shashwat Lal Das & sldas@tamu.edu & UIN: 321005607 \\ 
\end{tabular}
\end{center}

\vspace{0.2cm}

\begin{enumerate}

\item \textbf{Objective}

  We plan to develop a library in C++ which enables serialization and
  deserialization of data in a portable and easy to use fashion.

\item \textbf{Need for such a library}

  Writing data generated or manipulated by a program into a stream
  (serialization) is a common need experienced across various domains,
  such as machine learning (storing data sets modified or generated by
  the program), scientific programming, etc.

  Common techniques to solve this problem include writing data into
  XML files, or developing a unique way to perform serialization for
  every problem. In general, there does not appear to be a uniform way
  of performing this in a portable and efficient manner which works
  for most problems. Writing custom code for doing this is cumbersome
  and error prone.

\item \textbf{Requirements}

  As mentioned, portability and simplicity are fundamental
  requirements.
  
  Usually, the user needs to store not only plain old data types like
  integers, but structured data as well. Structured data may take the
  form of classes, structs, etc. with a specific layout, and the user
  should be able to serialize such data equally easily. To this end,
  the user defines the structure, and serialization should preserve
  this structure while storing the data. Deserialization should
  reconstruct this structure directly.

\item \textbf{Features Planned}
  
  Be able to store data of most common data types (int, string, etc.)
  in a compact form. The method of encoding may differ based on the
  data type.

  The user should be able to choose between human readable or binary
  representation. Naturally, human readable format would imply less
  emphasis on compactness.
  
  Include support for aggregates of data in the form of arrays and
  vectors (compulsorily). Whether support is added for other less
  common list types which may be more complex to store is not yet
  finalized.
  
    
\end{enumerate}

\end{document}
